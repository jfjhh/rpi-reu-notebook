\documentclass[handout,final]{beamer}

\usepackage{amsmath}
\usepackage{amssymb}
\usepackage{amsthm}
\usepackage{booktabs}
% \usepackage[figurename=(c),labelsep=space]{caption} % Hack
\usepackage{enumitem}
\usepackage{float}
\usepackage{geometry}
\usepackage{mathtools}
\usepackage{microtype}
\usepackage{nicefrac}
\usepackage{physics}
\usepackage{siunitx}
\usepackage{tikz}
\usepackage{xcolor}
\usepackage{pgffor}
\usepackage[orientation=landscape,size=custom,height=60.96,width=91.44,scale=1.0]{beamerposter}
\usepackage{adjustbox}
\usepackage{subcaption}

\graphicspath{{figs/}}

\usepackage[sfdefault,semibold]{libertine} % a bit lighter than Times--no osf in math
% \usepackage[libertine,vvarbb]{newtxmath}
\usepackage{eulervm}
\usepackage[scr=esstix,cal=boondoxo]{mathalfa}

\usetikzlibrary{arrows, shapes.misc, positioning, fit, calc}
\tikzset{%
  invisible/.style={opacity=0,text opacity=0},
  visible on/.style={alt=#1{}{invisible}},
  alt/.code args={<#1>#2#3}{%
    \alt<#1>{\pgfkeysalso{#2}}{\pgfkeysalso{#3}} % \pgfkeysalso doesn't change the path
  },
}

\usecolortheme[snowy]{owl}
\usefonttheme{professionalfonts}
% \usefonttheme{serif}
\setbeamersize{text margin left=0pt, text margin right=0pt}
\setbeamertemplate{blocks}[rounded][shadow=true]
\setbeamercolor{title}{fg=white,bg=blue} % Temporary; tweak me!
% \setbeamercolor{block title}{bg=blue!70, fg=white}
\setbeamercolor{block title}{bg=blue, fg=white}
\setbeamercolor{block body}{bg=black!9}
\setbeamerfont{block title}{series=\bfseries}

% Hack a margin into blocks.
\newenvironment{mblock}[1]{%
  \begin{block}{#1}%
    \begin{adjustbox}{minipage=\linewidth-3ex, margin=1.5ex, center}}%
{\end{adjustbox}\end{block}}

% It's a hack, but it works.
\title{%
  \parbox{0.02\textwidth}{\hspace{-2.25ex}\includegraphics[width=4.5ex]{nsf}\hfil}%
  \parbox{0.18\textwidth}{\small
    \textbf{Physics \textsc{Reu}: Summer 2020} \\
    Rensselaer Polytechnic Institute
  }
  \parbox{0.55\textwidth}{\Huge
    Understanding visual complexity with statistical physics
  }
  \parbox{0.20\textwidth}{\hfill
    Alex Striff \& Vincent Meunier
  }
}
\author{}
\date{}

\begin{document}
\begin{frame}[t]{}
  \vspace{-8pt}
  \begin{beamercolorbox}{}
    \maketitle
  \end{beamercolorbox}
  \vspace{-1.5in}
  \centering
  \tikzstyle{cclip}  = [transparent, circle, minimum size=5cm, outer sep=6pt]
  \tikzstyle{clabel} = [text opacity=1, inner sep=0pt, text=white, font=\bfseries]
  \begin{tikzpicture}[%
    visible on=<+->,
  ]
  \node at (0,0) {%
  \begin{tikzpicture}[
    visible on=<+->,
    x=7cm,
    y=7cm,
    cclip/.style = {transparent, circle, minimum size=5cm, outer sep=6pt},
    clabel/.style = {text opacity=1, inner sep=0pt, text=white, font=\bfseries, align=center}
    ]
    \node[visible on=<.(14)->, circle, fill=black!10, inner sep=0pt,
      outer sep=0pt, minimum size=19cm]
      (visual world) at (0,0) {};
    \node[visible on=<.(14)->, below=of visual world, inner sep=0pt, outer sep=0pt, text
      width=21cm, align=center] {%
        \textcolor{green}{We} cannot \textcolor{red}{directly perceive} the
        true nature of the world, but we can form models to
        \textcolor{orange}{indirectly perceive} nature. A model must represent
        nature, while still being something we can readily look at and use to
        guide thought. \textbf{What are good models of how nature looks to
        us?}
      };
    \begin{scope}[visible on=<.->]
      \node[cclip] (natural) at (120:1) {};
      \clip (natural) circle (2.5cm)
      node {\includegraphics[width=5cm]{tree-natural}};
      \node[clabel] at (natural) {Natural};
    \end{scope}
    \begin{scope}[visible on=<.(1)->]
      \node[cclip] (seen) at (0,0) {};
      \clip (seen) circle (2.5cm)
      node {\includegraphics[width=5cm]{tree-seen}};
      \node[clabel] at (seen) {Sight};
    \end{scope}
    \begin{scope}[visible on=<.(2)->]
      \node[cclip] (cartoon) at (0:1) {};
      \clip (cartoon) circle (2.5cm)
      node {\includegraphics[width=5cm]{tree-cartoon}};
      \node[clabel] at (cartoon) {Thought};
    \end{scope}
    \begin{scope}[visible on=<.(7)->]
      \node[cclip] (bitmap) at (-120:1) {};
      \clip (bitmap) circle (2.5cm)
      node {\includegraphics[width=5cm]{tree-bitmap}};
      \node[clabel] at (bitmap) {Model};
    \end{scope}
    \begin{scope}[->, line width=3pt]
      \begin{scope}[green]
        \draw[visible on=<.(1)->] (natural) to[bend right=15] node[below,sloped]{Look} (seen);
        \draw[visible on=<.(2)->] (seen) to[bend left=15] node[above]{Perceive} (cartoon);
        \draw[visible on=<.(3)->] (cartoon) to[bend left=15] node[below]{Reflect} (seen);
        \draw[visible on=<.(9)->] (bitmap) to[bend left=15] node[above,sloped]{Look} (seen);
      \end{scope}
      \begin{scope}[orange]
        \draw[visible on=<.(8)->] (natural) to[bend right=45] node[above,sloped,rotate=180]{Represent} (bitmap);
        \draw[visible on=<.(10)->] (seen) to[bend left=15] node[below,sloped]{Rep.} (bitmap);
        \draw[visible on=<.(11)->] (bitmap) to[bend right=45] node[above,sloped]{Learn} (cartoon);
        \draw[visible on=<.(12)->] (cartoon)  to[bend left=25] node[above,sloped]{Abstract} (bitmap);
        \draw[visible on=<.(13)->] (bitmap) to[bend left=25] node[above,sloped]{Generate} (natural);
      \end{scope}
      \begin{scope}[red]
        \draw[visible on=<.(4)->] (natural) to[bend left=25] node[above,sloped]{Omniscience} (cartoon);
        \draw[visible on=<.(5)->] (cartoon) to[bend right=45] node[above,sloped]{Physics} (natural);
        \draw[visible on=<.(6)->] (seen) to[bend right=15] node[above,sloped]{Psych.} (natural);
      \end{scope}
    \end{scope}
    \addtocounter{beamerpauses}{14}
  \end{tikzpicture}
  };

  \begin{scope}[%
    x=7cm,
    y=7cm,
    shift={($(visual world)+(0.333333\textwidth,0.25\textheight)$)},
  ]
    \node[circle, fill=black!10, inner sep=0pt,
      outer sep=0pt, minimum size=15cm]
      (visual complexity) at (0,0) {};
    \node[below=of visual complexity, inner sep=0pt, outer sep=0pt, text
      width=16cm, align=center] {%
        \textbf{Visual complexity} arises in many ways. Computational models of
        these phenomena are the bricks that build up the structure of an image.
      };
    \begin{scope}
      \node[cclip] (ifs) at (-135:0.625) {};
      \clip (ifs) circle (2.5cm)
      node {\includegraphics[height=5cm]{ifs}};
      \node[clabel,text=black,align=center] at (ifs) {Iterated \\ function \\ systems};
    \end{scope}
    \begin{scope}
      \node[cclip] (noise) at (45:0.625) {};
      \clip (noise) circle (2.5cm)
      node {\includegraphics[height=5cm]{perlin}};
      \node[clabel,align=center,text=black] at (noise) {Noise};
    \end{scope}
    \begin{scope}
      \node[cclip] (turbulence) at (135:0.625) {};
      \clip (turbulence) circle (2.5cm)
      node {\includegraphics[height=5cm]{turbulence}};
      \node[clabel,text=black,align=center] at (turbulence) {Turbulence};
    \end{scope}
    \begin{scope}
      \node[cclip] (fern) at (-45:0.625) {};
      \clip (fern) circle (2.5cm)
      node {\includegraphics[height=5cm]{fern}};
      \node[clabel,text=black,align=center] at (fern) {L-systems};
    \end{scope}
  \end{scope}

  \node[xshift=0.666667\textwidth,yshift=0.25\textheight] at (visual world) {%
  \begin{tikzpicture}[
    visible on=<+->,
    x=7cm,
    y=7cm,
    cclip/.style = {transparent, circle, minimum size=5cm, outer sep=6pt},
    clabel/.style = {text opacity=1, inner sep=0pt, text=white, font=\bfseries, align=center}
    ]
    \node[circle, fill=black!10, inner sep=0pt,
      outer sep=0pt, minimum size=13cm]
      (graphics) at (0,0) {};
    \node[below=of graphics, inner sep=0pt, outer sep=0pt, text
      width=18cm, align=center] {\vbox{%
        These methods are \textbf{widely used} to create graphics.
        \\\vspace{0.5\baselineskip}\small
        The electric sheep ``fractal flame'' screen saver is based on
          iterated function systems.
        Noise and turbulence are combined to procedurally generate
          textures for 3D models.
        L-systems are a component of \textit{SpeedTree}, which is the
          software used to create vegetation for movies like \textit{Avatar}.
      }};
    \begin{scope}
      \node[cclip] (sheep) at (120:0.5) {};
      \clip (sheep) circle (2.5cm)
      node {\includegraphics[height=5cm]{sheep}};
      \node[clabel] at (sheep) {Fractal \\ flame};
    \end{scope}
    \begin{scope}
      \node[cclip] (avatar) at (0:0.5) {};
      \clip (avatar) circle (2.5cm)
      node at (avatar) {\includegraphics[height=5cm]{avatar}};
      \node[clabel,text=black] at (avatar) {Vegetation};
    \end{scope}
    \begin{scope}
      \node[cclip] (mud) at (-120:0.5) {};
      \clip (mud) circle (2.5cm)
      node {\includegraphics[height=5cm]{mud}};
      \node[clabel] at (mud) {Procedural \\ Texture};
    \end{scope}
  \end{tikzpicture}
  };

  \node[xshift=0.5\textwidth,yshift=0\textheight] at (visual world) {%
  \begin{tikzpicture}[
    visible on=<+->,
    x=7cm,
    y=7cm,
    cclip/.style = {transparent, circle, minimum size=5cm, outer sep=6pt},
    clabel/.style = {text opacity=1, inner sep=0pt, text=white, font=\bfseries}
    ]
    \node[fill=black!10, inner sep=0pt,
      outer sep=0pt, minimum width=12.5cm, minimum height=6cm, rounded rectangle]
      (rfs) at (0,0) {};
    \node[below=of rfs, inner sep=0pt, outer sep=0pt, text
      width=16cm, align=center] {%
        \textbf{Receptive fields} of \textit{simple cells} in area V1 are
        modeled by space-time derivatives, and are related to smoothed
        derivatives of the light field entering the eye.
      };
    \begin{scope}
      \node[cclip] (rfmap) at (-180:0.425) {};
      \clip (rfmap) circle (2.5cm)
      node {\includegraphics[height=5cm]{rf-map}};
      \node[clabel,align=center] at (rfmap) {Map};
    \end{scope}
    \begin{scope}
      \node[cclip] (rfderiv) at (0:0.425) {};
      \clip (rfderiv) circle (2.5cm)
      node {\includegraphics[height=5cm]{rf-deriv}};
      \node[clabel,align=center] at (rfderiv) {Model};
    \end{scope}
  \end{tikzpicture}
  };

  % \node[xshift=0.5\textwidth,yshift=-0.25\textheight] (wlresults) at (visual world) {%
  \node[shift={($(visual world)+(0.5\textwidth,-0.25\textheight)$)}] (wlresults) {%
  \begin{tikzpicture}[%
    visible on=<+->,
    ]
    \node (wlerr) at (0,0) {\includegraphics[width=20cm]{wanglandau-bw-relerror}};
    \node[right=-2cm of wlerr] (wles) {\includegraphics[width=20cm]{wanglandau-gray-ES}};
  \end{tikzpicture}
  };
  
  \node[visible on=<.->,below=-0.5ex of wlresults,align=center,text width=40cm] {%
    For a notion of lost information in an image, we may perform simulations of
    \textbf{images as discrete-level systems}, where the
    \textcolor{blue}{energy} is the sum of the differences in pixel values. The
    \textit{Wang-Landau} algorithm was implemented for images, and confirmed to
    have the expected convergence properties on \textcolor{orange}{black or
    white (BW)} images (\emph{left}). Simulations for \num{1024} random
    grayscale images (\textit{right}) reveal how entropy scales with how
    different the pixel values are (average energy).
  };

  % \node[visible on=<.->,xshift=11.5cm,yshift=-1.5cm] at (wlresults) {%
  \node[visible on=<.->,shift={($(wlresults)+(11.5cm,-1.5cm)$)}] at (0,0) {%
  \tikzstyle{emph} = [fill=blue]
  \begin{tikzpicture}[line width=2pt, every node/.style={transform shape}]
    \node[left] at (1.25*1, 1 + 0.5) {\small$1$};
    \node[left] at (1.25*1, 5 + 0.5) {\small$M$};
    \node[below] at (1.25*1 + 0.5,1) {\small$1$};
    \node[below] at (1.25*4 + 0.5,1) {\small$N$};
    \fill[gray] (1.25*1,4) rectangle ++(1,1);
    \fill[emph] (1.25*1,3) rectangle ++(1,1);
    \fill[gray] (1.25*2,1) rectangle ++(1,1);
    \fill[emph] (1.25*2,5) rectangle ++(1,1);
    \fill[gray] (1.25*4,2) rectangle ++(1,1);
    \fill[emph] (1.25*4,4) rectangle ++(1,1);
    \draw[emph] (1.25*4 + 1, 2 + 0.5) -- ++(0.75,0);
    \draw[emph] (1.25*4 + 1, 4 + 0.5) -- ++(0.75,0);
    \draw[emph,<->] (1.25*4 + 1.5, 2 + 0.5) -- ++(0,2)
      node[midway,right]{$\Delta E_N = 2$};
    \node (ellipses) at (1.25*3 + 0.5, 3 + 0.5) {$\cdots$};
    \foreach \x in {1,2,4} {%
      \foreach \y in {1,...,5} {%
        \draw (1.25*\x,\y) rectangle ++(1,1);
      }
    }
  \end{tikzpicture}
  };

  % \node[x=1cm,y=1cm,xshift=0\textwidth,yshift=-0.333333\textheight] at (visual world) {%
  % \begin{tikzpicture}[%
  %   visible on=<+->,
  % ]
  \begin{scope}[xshift=0\textwidth,yshift=-0.333333\textheight]
    \node (entropy) {%
      $S
      = -\sum_{x \in \mathcal{X}} p(x) \ln p(x)
      = {\langle I \rangle}_X$
    };
    \node[below=-0.5ex of entropy] (kldiv) {%
      $D_{KL}(p \mathbin{\|} q)
      = \sum_{x \in \mathcal{X}} p(x) \ln \frac{p(x)}{q(x)}$
    };
    \node[below=-0.5ex of kldiv] (pfunc) {%
      $Z = \sum_{x \in \mathcal{X}} e^{-\beta E(x)}$
    };
    \node[fit=(entropy)(kldiv)(pfunc)] (ibox) {};
    \node[above=-0.5ex of ibox] {\textbf{Statistical Physics and Information Theory}};
  \end{scope}
  % \end{tikzpicture}
  % };

  \node[visible on=<.->,above=10ex of visual world,align=center,text width=35cm] {%
      Visual complexity is challenging to express. Here we have shown just three
      perspectives: those of computation, psychophysics, and of statistical
      physics. Many more remain to be explored. The complexity that we see is
      deeply connected to the complexity and variety of physical phenomena in
      our world.
  };
  \end{tikzpicture}
\end{frame}
\end{document}

