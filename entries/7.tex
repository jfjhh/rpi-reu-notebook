\documentclass[../notebook.tex]{subfiles}

\newcommand{\imsp}{\mathsf{Img}}

\begin{document}
\nbentry{June 4, 2020}{%
  Thermodynamic quantities for images from a microscopic model
}

Since we are thinking of images as statistical entities, what is the
corresponding microscopic model? Given such a model, what quantities do we
consider in thermal equilibrium, and how can we understand different ensembles?

\subsection{Quantum filled-site model (\textsc{fsm})}

\begin{defn}[\textsc{Fsm}]
  We define a lattice model corresponding to a \emph{ground image} $I_0$ as
  follows. Each pixel with value $k_0 \in K \subseteq \ZZ$ in the image
  corresponds to a \emph{site}, which is a discrete system with $K$
  \emph{levels}. The energy of level $k$ is $E_{k_0}(k) = \epsilon\abs{k -
    k_0}^r$.\footnote{If we want to consider colors, then $K$ is a metric space
  and we replace $k - k_0$ with the metric.} We usually have $r = 1$ or $2$. All
  of the image levels are occupied together by $N$ particles: \emph{pixons}
  (bosons) or \emph{crayons} (fermions). In the ground state, particles are
  evenly partitioned among all pixels. At each pixel, pixons will be at $k =
  k_0$ and crayons will fill the lowest levels.

  The maximum possible number of pixons at a site is $N_p = N$ and of crayons is
  $N_c = 1$. This is abbreviated with $\sigma = p$ or $c$ as $N_\sigma$.
  Consider (orthonormal and complete) occupation number states
  \[
    \ket{N_\sigma}
    \equiv \sum_{n=0}^{N_\sigma} \ket{n}
    \qand
    \ket{N_\sigma}^{\otimes\abs{K}\otimes N}
    = \bigotimes_{s = 0}^N \bigotimes_{k \in K} \ket{N_\sigma}
  \]
  (where order assigns occupation numbers to one-particle states), and define
  \begin{align}
    \ket{n_{ik},\, m_{i'k'}}
    &\equiv \ket{N_\sigma}^{\otimes\abs{K}\otimes(i-1)} \\
    &\otimes \ket{N_\sigma}^{\otimes(k - 1)}
    \otimes \ket{n}
    \otimes \ket{N_\sigma}^{\otimes(\abs{K} - k)} \\
    &\otimes \ket{N_\sigma}^{\otimes\abs{K}\otimes(i'-i)} \\
    &\otimes \ket{N_\sigma}^{\otimes(k' - 1)}
    \otimes \ket{m}
    \otimes \ket{N_\sigma}^{\otimes(\abs{K} - k')} \\
    &\otimes \ket{N_\sigma}^{\otimes\abs{K}\otimes(N-i')}
    \qq{(or $\ket{m_{i'k'},\, n_{ik}}$ if $i' < i$),}
  \end{align}
  which concentrates on the occupations of $\ket{i,\,k}$ and $\ket{i',\, k'}$.
  We may then express the interactions between different levels with the
  Hamiltonian
  \[
    \ham
    = \sum_{i,i' \in \ZZ_N} \sum_{k,k' \in K}
    \sum_{n=0}^{N_\sigma} \sum_{m=1}^{N_\sigma}
    H_{ii'kk'}\op{{(n+1)}_{ik},\, {(m-1)}_{i'k'}}{n_{ik},\, m_{i'k'}},
  \]
where $H_{iikk} = E_{k_0}(k)$ and the other elements may be varied for different
dynamics. The $H_{ii'kk'}$ terms describe the interaction between particles at
site $i'$, level $k'$ with particles at site $i$, level $k$.
\end{defn}

In particular, $H_{ii'kk'} = 0$ for $n \ne n'$ corresponds to isolated sites,
and $H_{ii'kk'} = 0$ for $k \ne k'$ corresponds to isolating levels from each
other but not from the same level at different sites. Combining both of these
conditions so that $H_{ii'kk'} = E_{k_0}(k)\delta_{ii'}\delta_{kk'}$ gives a
system where all of the image occupancy states are energy eigenstates. It may be
more clear to just consider this.

Usually the interactions at site $i$ with other sites are only allowed with
neighboring sites $\mathcal{N}_i \subseteq \ZZ_N$, so that $H_{ii'kk'} = 0$ for
$i' \notin \mathcal{N}_i \cup \{i\}$. For simplicity we often require that the
$H_{ii'kk'}$ are the same for $i' \in \mathcal{N}_i$.

\subsection{Observables and thermodynamic state variables}

Several observables of the $\textsc{fsm}$ are of interest:
\begin{itemize}
  \item \textbf{Pixel colors}. The occupations $n_k$ of different levels at a
    \emph{single} site induce a distribution on $K$. In equilibrium, the mean
    level
    \[
      \ev{k}
      \equiv \frac{\sum_{k \in K} k n_k}{\sum_{k \in K} n_k}
    \]
    should be near $k_0$, since the energy of a level is symmetric about $k_0$.
    As the temperature increases, so will the variance of the mean level. On the
    flip side, does \emph{varying} $k_0$ for many pixels quasistatically
    (changing the ground image) do work on the system? Yes, but is this
    consistent with what we expect?

  \item \textbf{Color distribution}. The net occupations $m_k$ of different
    levels across \emph{all} sites induce a distribution on $K$. In the special
    case of gray images (so levels are intensity), the entropy of the induced
    random variable is intensity entropy that we have studied previously. This
    is distribution is stationary when different levels cannot interact, but is
    it so at finite temperature?
    
  \item \textbf{Opacity}. The net occupancy of a \emph{site} could be connected
    to its opacity. In equilibrium, this should be similar across all sites.
    Then regions with \emph{no} particles during nonequilibrium processes make
    sense. The picture of a gas with fluctuating density that emits light comes
    to mind. When at maximum opacity, the gas in that region cannot be
    compressed further, and cannot accept more particles. The canonical density
    properties of a photon gas (like energy density) might be a good reason to
    choose the type of the particles to be bosonic (\emph{crayons}).

  \item \textbf{Number of pixels}. We could vary the total number of pixels
    different ways. One way is to have a continuous ground image, and choose
    different grid discretizations. Another way is to have a large ground image
    grid and vary the zoom level. It would be sensible to combine these sorts of
    transformations with pixel color transformations, since they include
    translation and rotation as special cases. This seems most most similar to
    varying the volume of a gas. Including opacity makes fast adiabatic piston
    motion volume changes like $V \mapsto 2V$ make sense.

  \item \textbf{Number of particles}. Depending on if we allow interactions
    between levels, it may be appropriate to consider chemical potentials.
    Either for all particles, or for each color. Could this be conjugate to
    opacity?

  \item \textbf{The usual}. Given that quasistatic transformation of the other
    quantities does work the way we expect, we can consider the usual response
    variables like heat capacities and compressibilities. There is also the
    thermodynamic entropy.
\end{itemize}

\end{document}

