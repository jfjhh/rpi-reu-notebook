\documentclass[../notebook.tex]{subfiles}

\begin{document}
\nbentry{June 26, 2020}{%
  Progress summary
}\label{sec:sum4}

This week, I improved my implementation of the Wang-Landau algorithm and
verified its correctness. I completed the convenience functions from before,
which allowed me to manage and combine thousands of simulations to characterize
the error in the density of states. For the same simulation parameters, my
relative error on the order of $10^{-4}$ in the density of states for image
systems is comparable to that of Landau ($10^{-3.5}$) for an Ising
ferromagnet.\footnote{\textsc{doi}:
\href{https://doi.org/10.1119/1.1707017}{10.1119/1.1707017}}
The level of the microcanonical entropy is now adjusted so that the sum over the
density of states is the total number of states. This prioritizes moderate
temperature correctness, but we may also set the level according to a ground
state, which increases correctness at low temperatures. I also calculated the
corresponding errors in thermodynamic quantities like heat capacity, which gave
consistent results.

Another batch of simulations was performed for random grayscale images. The
resulting densities of states were spread out over energies past half the
maximum energy for a black image, as expected. The energy, heat capacity, and
entropy all varied little across random images for temperatures which induce
energy fluctuations at the scale of one gray level, but showed significant
variance at intermediate temperatures. This reflects how the energy landscape
near the ground image is the same for all gray images, but varies as the
temperature becomes high enough to reach the bounds of allowed gray values.

Now that the Wang-Landau algorithm has proven itself to be a useful tool for
density estimation, we aim to use it and other methods to quantify the relative
information content of different aspects of vision for the second half of the
project.

\end{document}


