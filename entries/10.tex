\documentclass[../notebook.tex]{subfiles}

\newcommand{\imsp}{\mathsf{Img}}

\begin{document}
\nbentry{June 12, 2020}{%
  Progress summary
}\label{sec:sum2}

This week, I investigated different methods for obtaining thermodynamic
quantities from simulations. The issue is that quantities like entropy and the
Helmholtz free energy depend on global properties of the phase space (the
probability or density of a microstate), and thus cannot be constructed as
cumulants of microstates during a simulation. A related issue is the
improbability of ``tunneling'' across energy barriers when taking the usual
temperature-weighted steps.

These considerations motivate histogram-based methods, like the Wang-Landau
algorithm that I implemented. Instead of operating in the canonical ensemble, we
take a biased random walk on energies so that the result is a flat histogram of
visited energies. The density of states from this process may then be used to
compute a canonical partition function. Modifications where we keep a joint
density of states with respect to another variable make other ensembles
accessible.

Further progress with the Wang-Landau algorithm was slowed by a discrepancy in
the steps needed between Wang and Landau's results and mine for the 32 by 32
Ising ferromagnet. Their original paper claims a \SI{0.035}{\percent} average
error in the density of states after only \num{700000} total spin
flips.\footnote{\texttt{10.1103/PhysRevLett.86.2050}} This is far fewer than the
spin flips I needed, and another paper that looks at the scaling of the
tunneling time (spin flips to go from ground to anti-ground state) might
corroborate this.\footnote{\texttt{10.1103/PhysRevLett.92.097201}} Their
tunneling times are all well above \num{1720000} (an eighth of their
$\tau_\text{exact}$) for the same simulation. Success in the Wang-Landau
algorithm requires visiting all energies many times, so execution takes several
tunneling times. We are still trying to resolve this discrepancy, as well as
other vague details from the original paper, like the possible choice of energy
bins for continuous systems and unspecified edge behavior for energy intervals.

I have looked at other histogram methods like \textsc{wham}, as well as parallel
tempering (replica exchange \textsc{mcmc}). Both parallel tempering and the
Wang-Landau algorithm are attractive because they are easily parallelizable. I
have also come back to considering the \textsc{maxent}/\textsc{mem} image
reconstruction technique and the issues of entropy and feature representation
again.

\end{document}

