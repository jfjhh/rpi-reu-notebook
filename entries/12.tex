\documentclass[../notebook.tex]{subfiles}

\newcommand{\imsp}{\mathsf{Img}}

\begin{document}
\nbentry{June 16, 2020}{%
  Exact density of states for gray and \textsc{bw} images
}

In doing the future statistical simulations of images, we would like to know the
density of states $g(E)$ in our site-based model. This is the number of ways to
choose the $N$ pixel values $m_n$ from $0,\, \ldots,\, M$ so that
\[
  E
  = \sum_{n = 1}^N \abs{m_n - m_n^*},
\]
where $0 \le E \le MN$ and $m_n^*$ is the value of the ground image at pixel
$n$. While the solution to follow works for arbitrary values of $m_n^*$, the
results are simplest (no multinomial coefficients) if we consider $m_n^* = 0$ or
$M$ or $m_n^* = M/2$. That is, the image is all black or white (\textsc{bw}) or
gray. Then we have
\[
  E
  = \sum_{n = 1}^N g' m_n
\]
with $m_n = 0,\, \ldots,\, M$ and $g' = 1$ (\textsc{bw}) or $m_n = 0,\,
\ldots,\, M/2$ and $g' = 2$ (gray).\footnote{This gives $m = 0$ degeneracy 2,
which is not strictly correct but matters little.}
In that case, we'll consider \textsc{bw} for concreteness.

We encode the energy as an exponent, so that $g(E)$ is the coefficient of $x^E$
in the polynomial
\[
  p(x)
  \equiv {\qty(x^0 + x^1 + \cdots + x^M)}^N,
\]
since each inner sum represents different assignments of a pixel value and the
products will produce terms that represent all different combinations of values.
We then expand to find
\begin{align}
  p(x)
  &= {\qty(\frac{1 - x^{M+1}}{1 - x})}^N \\
  &= \sum_{k=0}^N \sum_{j=0}^\infty
  {(-1)}^k \binom{N}{k} x^{(M+1)k} \cdot {(-1)}^j \binom{-N}{j} x^j \\
  &= \sum_{k=0}^N \sum_{j=0}^\infty
  {(-1)}^k \binom{N}{k} \binom{N + j - 1}{j} x^{(M+1)k + j}.
\end{align}
The value of $j$ in the inner sum for $x$ to have power $E$ is $j = E - k(M+1)$,
so the coefficient of $x^E$ in $p(x)$ is
\begin{equation}
  \label{eq:imgdos}
  g(E)
  = \sum_{k=0}^N
  {(-1)}^k \binom{N}{k} \binom{N + E - k(M+1) - 1}{E - k(M+1)}.
\end{equation}
(Due to the stucture of the binomial coefficients, we may let the summation
extend over all $k \in \ZZ$ for further manipulation.)

\end{document}


