\documentclass[../notebook.tex]{subfiles}

\newcommand{\imsp}{\mathsf{Img}}

\begin{document}
\nbentry{June 14, 2020}{%
  ``Greedy'' painting-like pictures
}

We noticed that swapping neighboring pixels in the previous simulations produced
images that looked somewhat like paintings. There are other ways to make images
look like paintings, and this is one of them. We randomly draw a shape of fixed
size on the image. The color of the shape is the mean color of the image in the
rectangle where the shape is drawn. This is done many times, and the size of the
shapes gradually decreases. I call this ``greedy'' because choosing the mean
color is the locally optimal color. This method has the advantage of being
similar to the process of painting, which is why it is presented instead of the
usual ``oilify'' filters in image processing software. An improvement would try
to draw shapes that reflect the shapes of the subject matter, and perhaps add
other texture. These kinds of considerations might require more global
strategies.

\subfile{../python-notebooks/tex/greedycubism}

\end{document}



