\documentclass[../notebook.tex]{subfiles}

\newcommand{\imsp}{\mathsf{Img}}

\begin{document}
\nbentry{June 6, 2020}{%
  Progress summary (from beginning)
}\label{sec:sum1}

Over the last two weeks, I calculated some metrics on images and did some basic
simulations. The most important metric was the intensity entropy, which is used
as the ``entropy'' of an image in the maximum entropy method (\textsc{mem}) of
image reconstruction used by astronomers. This was calculated for a whole image
and locally in different regions of an image. On the topic of scaling, fractal
dimensions were explored. The box-counting dimension was computed for different
images, and the information dimension of the intensity distribution was
considered. I also did readings on probability theory and machine learning,
since the usual frequentist approach that experimental physicists take is not
applicable. Variants of Ising models were simulated for images, which led to the
postulation of a microscopic model for varying images (the \textsc{fsm}), which
is similar to a Hubbard model. The implications of this approach remain to be
explored.

\end{document}

