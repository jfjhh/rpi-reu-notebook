\documentclass[../notebook.tex]{subfiles}

\begin{document}
\nbentry{June 19, 2020}{%
  Progress summary
}\label{sec:sum3}

This week, I further prepared to study the thermodynamics of our image systems.
I derived the exact density of states for the all-black image, and the same
technique (generating functions) could be used with computer enumeration of
integer partitions to obtain the density of states for an arbitrary image
system. I also spent more time improving my implementation of the Wang-Landau
algorithm. It is now able to simulate arbitrary energy bins and divide different
energy intervals across multiple \textsc{cpu} cores and combine them back
together. The python code was made type-stable so that functions and the
simulation state can be \textsc{jit}ed by a \textsc{llvm}-based compiler
(\href{https://numba.pydata.org/}{Numba}). These two improvements increased the
speed of the simulations by more than a factor of 10. With some more tweaks and
code to automate the preparation of parallel kernels and manage results, we
should be able to easily perform these simulations for many parameter values
with high statistics.

\end{document}


