\documentclass[../notebook.tex]{subfiles}

\newcommand{\imsp}{\mathsf{Img}}

\begin{document}
\nbentry{June 3, 2020}{%
  Statistical Mechanics of Images
}

Given the qualitative success of the image-metric based Ising images, we
consider generalizations.

\begin{defn}
  An \emph{$N$-element image space} over metric spaces $(K, d)$ and $(P, a)$ is
  the space $\imsp = {(P \times K)}^N$. A corresponding \emph{image} is an
  element of $\imsp$.

  The space $P$ determines the spatial arrangement of the image, and is usually
  two-dimensional Euclidean space. We usually consider the subset of an image
  space where the $P$-coordinates are fixed, in a grid layout. The space $K$
  determines the qualities of an image at a point in $P$. This is usually a
  color or intensity space, and in practical applications is a machine integer
  like $128 \in \ZZ_8$.
\end{defn}

\begin{defn}
  An \emph{image system} on an image space $\imsp$ consists of a \emph{ground
  image} $I_0 \in \imsp$ and a \emph{dispersion relation} $E:\RR \to \RR$. This
  defines the \emph{energy} of an image $I$ as
  \[
    E(I)
    = \sum_{(p_0,\, k_0) \in I_0}
    \sum_{k \in \qty{k : (p_0,\, k) \in I}}
    E(d(k_0,\, k)).
  \]
\end{defn}

\begin{eg}
  For usual images in ${((\ZZ_n \times \ZZ_m) \times \ZZ_{256})}^N$, where $N =
  nm$ and the positions of $I_0$ and $I$ coincide (indexed by $i$ and $j$), we
  have
  \[
    E(I)
    = \sum_{i = 1}^n \sum_{j = 1}^m E(d(k_0^{ij},\, k^{ij}))
    = \sum_{i = 1}^n \sum_{j = 1}^m \epsilon\abs{k_0^{ij} - k^{ij}}^1,
  \]
  with typical choices of $E$ and $d$.
\end{eg}

In the binary case ($K = \ZZ_2$), we have $N$ independent two-level
systems.

\begin{eg}[Grayscale images]
  Consider a pixel of a ground grayscale image, with integer value $k_0 \in 0,\,
  \ldots,\, K-1$ for even $K$. There are then
  \[
    2g
    = 2\begin{cases}
      k_0, & k_0 < K/2 \\
      K - k_0 - 1, & \text{else}
    \end{cases}
  \]
  energy values that occur twice, and $K - 2g$ energy values that occur once
  (like $\abs{x}$ on an interval like $[-3, 5]$). Thus the partition function
  for this single pixel is
  \begin{align}
    \label{eq:Zpixel}
    Z_g
    &= \sum_{k=-g}^{K-g-1} e^{-\beta\epsilon\abs{k}}
    = 1
    + \sum_{k=1}^g e^{-\beta\epsilon k}
    + \sum_{k=1}^{K-g-1} e^{-\beta\epsilon k} \\
    &= 1
    + \frac{e^{-\beta g\epsilon}\qty(e^{\beta g\epsilon} - 1)}{e^{\beta\epsilon} - 1}
    + \frac{e^{-\beta (K-g-1)\epsilon}\qty(e^{\beta (K-g-1)\epsilon} -
    1)}{e^{\beta\epsilon} - 1}
  \end{align}
  and the partition function for the whole image is
  \begin{equation}
    \label{eq:Zimage}
    Z
    = \prod_{g = 0}^{-1 + K/2} Z_g^{NP(g)},
  \end{equation}
  where $NP(g)$ is the number of pixels in the ground image with the given
  $g$-value. We then see that
  \[
    \ln Z
    = \sum_{g = 0}^{-1 + K/2} NP(g) \ln Z_g
    = N\ev{\ln Z_g}_G,
  \]
  where $G$ is the random variable that takes the value $g$ with probability
  $P(g)$. It then follows that $\ev{E/N} = \ev{E_g}_G$ and $S/N = \ev{S_g}_G$ as
  usual for extensive variables.
\end{eg}

\end{document}


